\documentclass[a4paper]{article}
\usepackage{INTERSPEECH2018}

\title{
       Detecting Double-Talks (Overlapping Speech) in Conversations\\
       using Deep Learning
}

\name{
       Abdullah,
       Joachim K{\"o}hler,
       Michael Gref
}

\address{
       Fraunhofer IAIS
}

% FIXME: Set appropriate email addresses
\email{
       abdullah.motjuste@gmail.com,
       joachim.koehler@iais.fraunhofer.de,
       michael.gref@fraunhofer.de
}

\begin{document}

\maketitle

% ----------------------------------------------------------------------------------------- UTILS -
\newcommand{\outline}[1]{}  % outline notes that will not be exported.
\newcommand{\widom}[1]{}  % gudelines from https://cs.stanford.edu/people/widom/paper-writing.html

% -------------------------------------------------------------------------------------- ABSTRACT -
\begin{abstract}
\widom{
       - the problem,
       - the approach and solutions
       - the main contributions of the paper.
       - little if any background and motivation. address the experts.
       - factual but comprehensive.
}

We present a deep learning based approach for detecting overlapping speech which occur naturally during normal conversations (aka double-talk).
We evaluate appropriately training a Deep Convolutional Neural Nectwork (DCNN) in a supervised manner for the purpose,
which involves \ldots

\end{abstract}

% ----------------------------------------------------------------------------------- INDEX TERMS -
\noindent\textbf{Index Terms}:
overlapping-speech detection,
speech segmentation,
deep convolutional neural network,
re-balancing data,

% ---------------------------------------------------------------------------------- INTRODUCTION -
\section{Introduction}
\widom{
       - What is the problem?
       - Why is it interesting and important?
       - Why is it hard? (E.g., why do naive approaches fail?)
       - Why hasn't it been solved before? Or,
       - What's wrong with other solutions? How does mine differ?
       - What are the key components of my approach and results?
       - Any specific limitations of my approach?
       - Short yet detailed enough `Related Works` near the end, or make it as Section 2.
       - `Summary of Contributions` as the final para or subsection.
              - List the major contributions in bullet form,
              - Mention in which sections they can be found, doubling up as an outline.
}
\outline{
       1 [X] Double-Talks occur frequently & naturally during normal conversations.
       2 [ ] This is different from the Cocktail Party Problem.
       1 [X] Conversation Analysis / Interactional Linguistics is very interested in it.
       3 [ ] Can be a metric of ad-hoc relationships b/w participants, culture, etc.
       2 [X] Manual annotation is very expensive.

       1 [ ] Figure with segment lengths
       1 [X] They are frequent, but very short-lived, typical (median) duration of \leq{1.5} sec.
       1 [ ] The classes are heavily imbalanced, especially so against DT.
       2 [ ] Agrees with theories that a 1-spk speaks most of the time w/ min. gaps or overlaps.
       1 [ ] Automatic detection with high resolution is important, but extremely challenging.

       1 [ ] Most of existing speech technology work done on this problem is in Speaker Diarization.
       1 [ ] Diarzn is done purely using the acoustic data, and unsupervised.
       1 [ ] Most Diarzn are not designed for more than 1 simultaneous speakers.
       1 [ ] Motivated to improve SOTA performance, and has been termed as their Achilles' Heel.
       1 [ ] Diarzn based pre-processing for ASR that can't handle DT makes ASR suffer as well.
       3 [ ] And still does, even with DL-based leaps in conversational ASR.
       3 [ ] The errors are less obvious in ASR wrt WER as most overlaps are single word long.

       3 [ ] Ideal Diarzn is perfect for Conv-Analysts.
       1 [ ] Most proposals train a model-based standalone DT detection system.
       1 [ ] The detections are then used to exclude or attribute second speaker (no details).
       1 [ ] Other approaches \ldots (page 10 of thesis; no details)
       1 [ ] We are not concerned with handling DT, only detecting them, for the Conv-Analysts.

       1 [ ] Almost all dedicated DT-system approaches use a GMM-HMM based approach for 3 classes.
       1 [ ] There is also one with LSTM-HMM \ldots
       1 [ ] The performances are not really that good, leading to the claim of extreme difficulty.
       1 [ ] There is apparently a very steep trade-off between precision and recall.

       1 [ ] Previous works have worked on finding the right acoustic features.
       1 [ ] MFCCs are not enough.
       1 [ ] For mono-aural setting \ldots (main related works)
       1 [ ] There have been approaches using multi-mic, but they're not available in all situations

       1 [ ] There have also been other recent proposals, but only on artificially overlapped data.
       1 [ ] (Pyknogram rant.)
       1 [ ] (That CNN paper from Joachim.)
       1 [ ] (Why training & evaluating on artificial data should be done carefully.)

       1 [ ] Lack of big & accurate dataset is an issue, and perhaps hence artificial ones is used.
       1 [ ] We use Fisher Corpus for its size and natural overlaps. (more in appropriate section).
       1 [ ] We also limit ourselves to the mono-aural setting, so that the results are versatile.

       1 [ ] Finally, we explore using DCNN on reltvy. low-level FBanks to avoid feature engg.
       1 [ ] DL is powerful. DCNN is used as a feature extractor in many scenarios w/ great results.
       3 [ ] DL in unsupervised Diarzn is still an active area of research. (w/ summary?)
       1 [ ] DL for supervised DT-detection is straightforward to comprehend, but \ldots
       1 [ ] The severe class-imbalance makes it important to train the n/w properly.
       1 [ ] We discuss re-balancing the dataset in its own section.
       1 [ ] The proposed DCNN based classifier's preds. are noisy.
       1 [ ] We use Viterbi algo to smooth-out the predictions based on long-term temporal patterns.
       1 [ ] We report results from exhaustive empircal results on which strategies worked out.
       1 [ ] This includes brief observations on failure condn. of the DCNN, & total failure of DNN.
       1 [ ] leading to the conclusion that DL is a worthwhile direction to take,
       2 [ ] including the use of RNNs to fix the long-term pattern recognition.
}

% TODO: Figure out how to cite the author's names automatically

Overlapping speech, also interchangeably referred to as double-talk,
occurs when more than one speakers speak simultaneously at a given instant of time.
This is a very common occurrence in spontaneous conversations where other participants make an utterance while a speaker is already speaking with many possible intentions,
including non-competitive acknowledgement like “mhm" or reaction like laughing,
or competitive interruption like when having misjudged their turn to speak.
% In fact, the occurrence of double-talks can add to the naturalness of a conversation \cite{shriberg_spontaneous_2005,NenkovaHighFrequencyWord2008},
% and
Its occurrence
is a subject of interest for Conversation Analysis in studying the turn-taking management done by the participants in a conversation.
% ad-hoc mechanisms by which particpants manage turn-taking during a conversation.
While the frequency and duration of overlaps can vary depending on the situation,
overlaps are quite frequent and characteristically brief (predominently smaller than 1\,second)
during normal, spontaneous conversations.  % TODO: T [fig] for dt-distribution
The characteristically small duration makes precisely annotating them a very expensive manual endeavor,
while attempts to automate the process have found it to be an extremely challenging task.

The flagship scenario of most automated speech technologies is of spontaneous conversations,
and the presence of double-talks is often detrimental to their perfomance.
Speaker diarization systems, whose goal is to determine `who spoke when' in a conversation,
are penalized when they miss additional speakers in segments with overlapping speech.
In the increasingly better state-of-the-art performance of such systems,
this penalty has come to account for a major portion of the errors that remain,
so much so that Anguera et al. claim overlapping speech situations to be the `Achilles heel' of speaker diarization systems when applied to recordings of meetings \cite{anguera_speaker_2012}.
Many other speech technologies (e.g. Automatic Speech Recognition) rely on speaker homogenous segmentations produced by a speaker diarization system,
and have been reported to suffer from degraded performance in situations of double-talks \cite{cetin_speaker_2006}.

Perhaps consequently, most previous attempts at detecting natural double-talks in conversations have been made in lieu of improving speaker diarization systems.
The most successful approaches propose employing a dedicated and purely acoustic overlap detection system.
The problem has most commonly been formulated as performing frame-wise classification for the presence of either
zero (i.e. silence), one, or more than one simultaneously active speakers,
and the solutions have been implemented in a GMM-HMM based framework while engineering different combinations of acoustic features.  % TODO: T [ref] for all the main ovl in diarzn studies
In general, using additional features from blah, blah, blah, or blah % TODO: T [text] summarize other features
have been found to improve overlap detection over using only spectral features.
Geiger et al. blah blah blah  % TODO: T [text] summarize LSTM-HMM approach.
% TODO: T [search] sweep any new DL-based overlap detection in natural conversations.
The problem of overlap detection has remained unsolved
and continues to present a steep trade-off between precision and recall of the system.

The most potent source of challenges in detecting natural double-talks is rooted in the inherent imbalance between the three classes.
It can be seen in Table X that in spontaneous conversations,  % TODO: T [tbl] Add table of segment and duration proportions.
while the individual occurrences of natural overlaps constitute a significant proportion of the total number of speaker-homogenous segmentations,
due to their predominently small duration, they account for the smallest proportion of individual frames.
Some studies have attempted to solve for this issue by training their proposed systems on artificially overlapped speech.  % TODO: T [ref, search] pyknogram paper, and the recent one, and any others
Shokouhi et al. reported better results when using their proposed Pyknograms as acoustic features for detecting overlaps under different noise conditions,
but their evaluations showed a discouraging dip in performance when the overlaps were less than 2\,seconds long \cite{shokouhi_teager_2017}.
In one of the recent proposals using a Deep Learning based method,
Andrei et al. reported achieving one of the better results in detecting overlaps of 500, 100 and 25\,millisecond \textit{window durations} \cite{AndreiDetectingOverlappedSpeech2017}
(it is unclear from the paper if this is also the temporal resolution (\textit{hop-size}) at which the predictions were made)
when MFCCs were combined with other acoustic features as inputs to a convolutional neural network.
However, no evaluations of such systems have been reported on real conversations and,
for reasons discussed in Section X.x, % TODO: T [ref] to section discussing artificial data
the expectations may not be encouraging to take this route for solving the imbalance problem.

For our study, we have developed and evaluated our proposed system on naturally occurring double-talks in real telephonic conversations of the Fisher Corpus.  % TODO: T [ref] fisher paper, link
We have limited ourselves to work with mono-aural audio recorded at a sample rate of 8000\,Hertz,
in part forced by our choice of the dataset,
and in part motivated by our desire for the system to be applicable to almost all recording conditions.
Our system uses a Deep Convolutional Neural Network (DCNN) based automatic feature extractor and classifier working only on log-mel-spectrograms as inputs,
and hence avoids the need of previously prevelant feature engineering necessary for the task.
We present our proposals and systematic evaluation of different training strategies for solving the class imbalance issue and other inherent challenges posed by the problem.
Furthermore, we also evaluate the imporvements to our DCNN achieved by temporally smoothing its potentially noisy predictions
by using the Viterbi algorithm that incorporates longer-term temporal patterns that our DCNN is not able to learn on its own.

% -------------------------------------------------------------------------------------- APPROACH -
\section{Approach}
Our system performs frame-wise classification of the extracted acoustic features into three classes based on the number of simultaneously speaking speakers:
zero (silence), one, or more than one (overlap).
For this purpose, our decision for investigating deep learning methods, and DCNNs in particular,
was motivated by the desire to avoid the heavy feature engineering prevelant in previous works.
Recent ground-breaking results achieved by DCNNs have been attributed to their capability of automatically learning the appropriate features for a task from almost raw inputs.
For our approach, we used 64\,dimensional $\text{log}_{10} \text{-scaled}$ melspectrograms (filter-banks)
extracted every 10\,milliseconds over a window of 32\,milliseconds (equivalent to 256 FFT-bins for 8000\,Hertz audio),
which have been shown to achieve better performance than MFCCs and pure spectrograms in various deep learning based acoustic models.  % TODO: T [ref] papers about fbank vs mfcc
For the purpose of reducing the mismatch between training and testing conditions, and as a crude noise reduction method,
we applied cepstral mean normalization on the extracted features.
However, we performed this on a chunk-by-chunk basis,
where the mean vector of a contiguous chunk of audio of roughly 2.5\,minutes duration was used to center all the vectors of that chunk.
This avoids the reliability issues with utterance level normalization by using long enough chunks,
while also avoiding impact of unrelated outliers if it were done over the entire datatset.
Additional variance normalization was not found to be helpful,
which is in line with other proposals using deep architectures.  % TODO: T [ref] against variance normalization in DL

The particular architecture of our DCNN is a heavily simplified version of VGG-net (Table X),  % TODO: T [ref] VGG-net
and employs ReLU function for all activations except the final output layer's.
Dropout and batch-normalization layers are used for regualization,
and all trainings were performed using Adamax to optimize the categorical cross-entropy with parameters as per the original paper.  % TODO: T [ref] adamax paper
When constructing the input to our DCNN, we add 10 frames each from immediately before and after a given frame to be classified (see inputs in Table X)  % TODO: T [tbl] DCNN arch
with the goal to provide additional contextual information to the classifier.
This limits the number of convolutional blocks in our architecture to when the number of temporal dimensions left is reduced to one.
Our preliminary experiments showed that using fewer number of contextual frames gave relatively worse results,
while larger values came at significant computational cost.
We fixed our feature extraction pipeline and the DCNN's architecture for all our experiments in order to isolate the variables for the different training strategies discussed in later sections.

% ---------------------------------------------------------------------- TACKLING CLASS IMBALANCE -
\section{Tackling Class Imbalance}
While deep learning methods are extremely powerful,
they suffer from similar degradation in performance as classic approaches when the target classes are severely imbalanced.
There have been relatively few systematic studies to solve this problem in the context of deep learning.
Of ones that do exist, almost all have concentrated on image classification related tasks.  % TODO: T [ref,search] DL with imbalance
Their proposals are often not so straightforward to apply on speech signals.
We discuss these approaches in the following sections in the context of detecting overlapping speech.

% --------------------------------------------------------------------------------------- DATASET -
\subsection{Dataset}
\outline{
       1 [ ] DL requires large number of examples for better generalization.
       1 [ ] AMI, etc. are not large enough
       1 [ ] They have also been reported to be inconsistently annotated, and suffer from less than ideal recording conditions (zelenak, andrei_cnn)
       1 [ ] The choice has been made by other studies to create artificial overlaps.
       1 [ ] some use too clean a base dataset to create overlaps, and such systems fail on real conversations.
       1 [ ] this could be solved for by adding additional noise during training.
       1 [ ] Even then, it is not likely that the generated overlaps will match real ones.
       1 [ ] There are particular utterences, intonation changes, prosody, etc. factors that often occur in regions of overlaps (that paper)
       1 [ ] These would be very difficult to replicate in artificially created data.

       1 [ ] we deliberately chose Fisher Corpus 1.
       1 [ ] it is a large dataset, and the telephone based conversations have natural overlaps.
       1 [ ] We only worked with ~200 hours for training and validation, ~90 hours for testing.
       3 [ ] there is more overlap than the entire size of some dataset usage that has been reported.
       1 [ ] The size of data available also affords certain other decisions discussed in subsequent sections.

       1 [ ] However, it imposes some limitations. Our system's result is only strictly applicable to <= 2 speaker telephone based conversations
       1 [ ] But more than 2 speakers is very rare
       1 [ ] the telephone recording limits us to use 8khz, which is not ideal.
       1 [ ] but it is the least common denominator scenario, and other audios can be downsampled trivially.
       2 [ ] in fact, restricting the mel-scale to a fixed value (64 in our case), the down-sampling might not be necessary, as long as features are normalized.
       1 [ ] technically, the recordings have two speaker specific channels, but we merged them trivially using LibROSA.
       1 [ ] lastly, the dataset is not exactly high resolution and precise, sometimes missing gaps, leading to weird peaks in distribution, and a long tail.
       1 [ ] This could impact reliability of evaluation, and we observed the system to be better than the available annotations, helped by the large dataset.
       1 [ ] however, since we have independent speaker channels, we could re-annotate the data with a better SAD and retrain. We did not do this for our study.
}
Nevertheless, deep learning based methods necessitate a large training dataset for better generalizability of the learned models.
This requirement becomes even more important due to the disadvantaged proportion of double-talks in the imbalanced classes.


% ---------------------------------------------------------------------------------- BIBLIOGRAPHY -
\bibliographystyle{IEEEtran}
\bibliography{dt-paper}
\end{document}
