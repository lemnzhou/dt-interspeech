\documentclass[a4paper]{article}
\usepackage{INTERSPEECH2018}

\title{
       Detecting Double-Talks (Overlapping Speech) in Conversations\\
       using Deep Learning
}

\name{
       Abdullah,
       Joachim K{\"o}hler,
       Michael Gref
}

\address{
       Fraunhofer IAIS
}

% FIXME: Set appropriate email addresses
\email{
       abdullah.motjuste@gmail.com,
       joachim.koehler@iais.fraunhofer.de,
       michael.gref@fraunhofer.de
}

\begin{document}

\maketitle

% ----------------------------------------------------------------------------------------- UTILS -
\newcommand{\outline}[1]{}  % outline notes that will not be exported.
\newcommand{\widom}[1]{}  % gudelines from https://cs.stanford.edu/people/widom/paper-writing.html

% -------------------------------------------------------------------------------------- ABSTRACT -
\begin{abstract}
\widom{
       - the problem,
       - the approach and solutions
       - the main contributions of the paper.
       - little if any background and motivation. address the experts.
       - factual but comprehensive.
}

We present a deep learning based approach for detecting overlapping speech which occur naturally during normal conversations (aka double-talk).
We evaluate appropriately training a Deep Convolutional Neural Nectwork (DCNN) in a supervised manner for the purpose,
which involves \ldots

\end{abstract}

% ----------------------------------------------------------------------------------- INDEX TERMS -
\noindent\textbf{Index Terms}:
overlapping-speech detection,
speech segmentation,
deep convolutional neural network,
re-balancing data,

% ---------------------------------------------------------------------------------- INTRODUCTION -
\section{Introduction}
\widom{
       - What is the problem?
       - Why is it interesting and important?
       - Why is it hard? (E.g., why do naive approaches fail?)
       - Why hasn't it been solved before? Or,
       - What's wrong with other solutions? How does mine differ?
       - What are the key components of my approach and results?
       - Any specific limitations of my approach?
       - Short yet detailed enough `Related Works` near the end, or make it as Section 2.
       - `Summary of Contributions` as the final para or subsection.
              - List the major contributions in bullet form,
              - Mention in which sections they can be found, doubling up as an outline.
}
\outline{
       1 [X] Double-Talks occur frequently & naturally during normal conversations.
       2 [ ] This is different from the Cocktail Party Problem.
       1 [X] Conversation Analysis / Interactional Linguistics is very interested in it.
       3 [ ] Can be a metric of ad-hoc relationships b/w participants, culture, etc.
       2 [X] Manual annotation is very expensive.

       1 [ ] Figure with segment lengths
       1 [X] They are frequent, but very short-lived, typical (median) duration of \leq{1.5} sec.
       1 [ ] The classes are heavily imbalanced, especially so against DT.
       2 [ ] Agrees with theories that a 1-spk speaks most of the time w/ min. gaps or overlaps.
       1 [ ] Automatic detection with high resolution is important, but extremely challenging.

       1 [ ] Most of existing speech technology work done on this problem is in Speaker Diarization.
       1 [ ] Diarzn is done purely using the acoustic data, and unsupervised.
       1 [ ] Most Diarzn are not designed for more than 1 simultaneous speakers.
       1 [ ] Motivated to improve SOTA performance, and has been termed as their Achilles' Heel.
       1 [ ] Diarzn based pre-processing for ASR that can't handle DT makes ASR suffer as well.
       3 [ ] And still does, even with DL-based leaps in conversational ASR.
       3 [ ] The errors are less obvious in ASR wrt WER as most overlaps are single word long.

       3 [ ] Ideal Diarzn is perfect for Conv-Analysts.
       1 [ ] Most proposals train a model-based standalone DT detection system.
       1 [ ] The detections are then used to exclude or attribute second speaker (no details).
       1 [ ] Other approaches \ldots (page 10 of thesis; no details)
       1 [ ] We are not concerned with handling DT, only detecting them, for the Conv-Analysts.

       1 [ ] Almost all dedicated DT-system approaches use a GMM-HMM based approach for 3 classes.
       1 [ ] There is also one with LSTM-HMM \ldots
       1 [ ] The performances are not really that good, leading to the claim of extreme difficulty.
       1 [ ] There is apparently a very steep trade-off between precision and recall.

       1 [ ] Previous works have worked on finding the right acoustic features.
       1 [ ] MFCCs are not enough.
       1 [ ] For mono-aural setting \ldots (main related works)
       1 [ ] There have been approaches using multi-mic, but they're not available in all situations

       1 [ ] There have also been other recent proposals, but only on artificially overlapped data.
       1 [ ] (Pyknogram rant.)
       1 [ ] (That CNN paper from Joachim.)
       1 [ ] (Why training & evaluating on artificial data should be done carefully.)

       1 [ ] Lack of big & accurate dataset is an issue, and perhaps hence artificial ones is used.
       1 [ ] We use Fisher Corpus for its size and natural overlaps. (more in appropriate section).
       1 [ ] We also limit ourselves to the mono-aural setting, so that the results are versatile.

       1 [ ] Finally, we explore using DCNN on reltvy. low-level FBanks to avoid feature engg.
       1 [ ] DL is powerful. DCNN is used as a feature extractor in many scenarios w/ great results.
       3 [ ] DL in unsupervised Diarzn is still an active area of research. (w/ summary?)
       1 [ ] DL for supervised DT-detection is straightforward to comprehend, but \ldots
       1 [ ] The severe class-imbalance makes it important to train the n/w properly.
       1 [ ] We discuss re-balancing the dataset in its own section.
       1 [ ] The proposed DCNN based classifier's preds. are noisy.
       1 [ ] We use Viterbi algo to smooth-out the predictions based on long-term temporal patterns.
       1 [ ] We report results from exhaustive empircal results on which strategies worked out.
       1 [ ] This includes brief observations on failure condn. of the DCNN, & total failure of DNN.
       1 [ ] leading to the conclusion that DL is a worthwhile direction to take,
       2 [ ] including the use of RNNs to fix the long-term pattern recognition.
}

% TODO: Figure out how to cite the author's names automatically

Overlapping speech, also referred to as double-talk,
occurs when more than one speakers speak simultaneously at a given instant of time.
This is a very common occurrence in spontaneous conversations where other participants make an utterance while a speaker is already speaking for many possible reasons,
including non-competitive acknowledgements like “mhm" or reactions like laughing,
or competitive interruptions like when having misjudged their turn to speak.
In fact, the occurrence of double-talks can add to the naturalness of a conversation \cite{shriberg_spontaneous_2005,NenkovaHighFrequencyWord2008},
and is a subject of interest in the area of Conversation Analysis while studying the ad-hoc mechanisms by which particpants manage turn-taking during a conversation.
While the frequency and duration of overlaps can vary depending on the situation,
overlaps are quite frequent and characteristically brief (predominently smaller than 1\,second)
during normal, spontaneous conversations.  % TODO: T [fig] for dt-distribution
The characteristically small duration makes annotating them precisely a very expensive manual endeavor for Conversation Analysis,
while attempts to automate the process have found it to be an extremely challenging task.

The flagship scenario of most automated speech technologies is of spontaneous conversations,
and the presence of double-talks is often detrimental to the performance of such technologies when not appropriately handled.
Speaker diarization systems, whose goal is to determine `who spoke when' in a recording with more than one speakers,
are penalized when they miss additional speakers in segments with overlapping speech.
In the increasingly better state-of-the-art performance of such systems,
this penalty has come to account for a major portion of the errors that remain,
so much so that Anguera et al. claim overlapping speech situations to be the `Achilles heel' of speaker diarization systems when applied to recordings of meetings \cite{anguera_speaker_2012}.
Many other speech technologies (e.g. Automatic Speech Recognition) rely on speaker homogenous segmentations produced by a speaker diarization system in a pre-processing step,
and hence also suffer from degraded performance in situations of missed double-talks \cite{cetin_speaker_2006}.

Perhaps consequently, most previous attempts at detecting natural double-talks in conversations have been made in lieu of improving speaker diarization systems.
The most successful approaches propose employing a dedicated and purely acoustic overlap detection system.
The problem has most commonly been formulated as performing frame-wise classification for the presence of either zero (i.e. silence), one, or two (or more) currently active speakers,
and implemented in a GMM-HMM based framework while experimenting with different combinations of acoustic features.  % TODO: T [ref] for all the main ovl in diarzn studies
In general, using additional features from blah, blah, blah, or blah % TODO: T [text] summarize other features
have been found to improve overlap detection over using only spectral features.
Geiger et al. blah blah blah  % TODO: T [text] summarize LSTM-HMM approach.
% TODO: T [search] sweep any new DL-based overlap detection in natural conversations.
The problem of overlap detection has remained unsolved
and continues to present a steep trade-off between precision and recall of the system.

The most potent source of challenge in detecting overlaps is rooted in the inherent imbalance between the three classes.
It can be seen in Table X that in spontaneous conversations,  % TODO: T [tbl] Add table of segment and duration proportions.
while the individual occurrences of natural overlaps constitute a significant proportion of the total number of speaker-homogenous segmentations,
due to their predominently small duration, they account for the smallest proportion of individual frames.
% artificial overlaps


For sanity, let us refer to the paper \cite{rabiner_tutorial_1989} without the author's name.

% ---------------------------------------------------------------------------------- BIBLIOGRAPHY -
\bibliographystyle{IEEEtran}
\bibliography{dt-paper}
\end{document}
